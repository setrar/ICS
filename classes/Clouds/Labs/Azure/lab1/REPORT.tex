\documentclass[11pt]{article}

    \usepackage[breakable]{tcolorbox}
    \usepackage{parskip} % Stop auto-indenting (to mimic markdown behaviour)
    

    % Basic figure setup, for now with no caption control since it's done
    % automatically by Pandoc (which extracts ![](path) syntax from Markdown).
    \usepackage{graphicx}
    % Keep aspect ratio if custom image width or height is specified
    \setkeys{Gin}{keepaspectratio}
    % Maintain compatibility with old templates. Remove in nbconvert 6.0
    \let\Oldincludegraphics\includegraphics
    % Ensure that by default, figures have no caption (until we provide a
    % proper Figure object with a Caption API and a way to capture that
    % in the conversion process - todo).
    \usepackage{caption}
    \DeclareCaptionFormat{nocaption}{}
    \captionsetup{format=nocaption,aboveskip=0pt,belowskip=0pt}

    \usepackage{float}
    \floatplacement{figure}{H} % forces figures to be placed at the correct location
    \usepackage{xcolor} % Allow colors to be defined
    \usepackage{enumerate} % Needed for markdown enumerations to work
    \usepackage{geometry} % Used to adjust the document margins
    \usepackage{amsmath} % Equations
    \usepackage{amssymb} % Equations
    \usepackage{textcomp} % defines textquotesingle
    % Hack from http://tex.stackexchange.com/a/47451/13684:
    \AtBeginDocument{%
        \def\PYZsq{\textquotesingle}% Upright quotes in Pygmentized code
    }
    \usepackage{upquote} % Upright quotes for verbatim code
    \usepackage{eurosym} % defines \euro

    \usepackage{iftex}
    \ifPDFTeX
        \usepackage[T1]{fontenc}
        \IfFileExists{alphabeta.sty}{
              \usepackage{alphabeta}
          }{
              \usepackage[mathletters]{ucs}
              \usepackage[utf8x]{inputenc}
          }
    \else
        \usepackage{fontspec}
        \usepackage{unicode-math}
    \fi

    \usepackage{fancyvrb} % verbatim replacement that allows latex
    \usepackage{grffile} % extends the file name processing of package graphics
                         % to support a larger range
    \makeatletter % fix for old versions of grffile with XeLaTeX
    \@ifpackagelater{grffile}{2019/11/01}
    {
      % Do nothing on new versions
    }
    {
      \def\Gread@@xetex#1{%
        \IfFileExists{"\Gin@base".bb}%
        {\Gread@eps{\Gin@base.bb}}%
        {\Gread@@xetex@aux#1}%
      }
    }
    \makeatother
    \usepackage[Export]{adjustbox} % Used to constrain images to a maximum size
    \adjustboxset{max size={0.9\linewidth}{0.9\paperheight}}

    % The hyperref package gives us a pdf with properly built
    % internal navigation ('pdf bookmarks' for the table of contents,
    % internal cross-reference links, web links for URLs, etc.)
    \usepackage{hyperref}
    % The default LaTeX title has an obnoxious amount of whitespace. By default,
    % titling removes some of it. It also provides customization options.
    \usepackage{titling}
    \usepackage{longtable} % longtable support required by pandoc >1.10
    \usepackage{booktabs}  % table support for pandoc > 1.12.2
    \usepackage{array}     % table support for pandoc >= 2.11.3
    \usepackage{calc}      % table minipage width calculation for pandoc >= 2.11.1
    \usepackage[inline]{enumitem} % IRkernel/repr support (it uses the enumerate* environment)
    \usepackage[normalem]{ulem} % ulem is needed to support strikethroughs (\sout)
                                % normalem makes italics be italics, not underlines
    \usepackage{soul}      % strikethrough (\st) support for pandoc >= 3.0.0
    \usepackage{mathrsfs}
    

    
    % Colors for the hyperref package
    \definecolor{urlcolor}{rgb}{0,.145,.698}
    \definecolor{linkcolor}{rgb}{.71,0.21,0.01}
    \definecolor{citecolor}{rgb}{.12,.54,.11}

    % ANSI colors
    \definecolor{ansi-black}{HTML}{3E424D}
    \definecolor{ansi-black-intense}{HTML}{282C36}
    \definecolor{ansi-red}{HTML}{E75C58}
    \definecolor{ansi-red-intense}{HTML}{B22B31}
    \definecolor{ansi-green}{HTML}{00A250}
    \definecolor{ansi-green-intense}{HTML}{007427}
    \definecolor{ansi-yellow}{HTML}{DDB62B}
    \definecolor{ansi-yellow-intense}{HTML}{B27D12}
    \definecolor{ansi-blue}{HTML}{208FFB}
    \definecolor{ansi-blue-intense}{HTML}{0065CA}
    \definecolor{ansi-magenta}{HTML}{D160C4}
    \definecolor{ansi-magenta-intense}{HTML}{A03196}
    \definecolor{ansi-cyan}{HTML}{60C6C8}
    \definecolor{ansi-cyan-intense}{HTML}{258F8F}
    \definecolor{ansi-white}{HTML}{C5C1B4}
    \definecolor{ansi-white-intense}{HTML}{A1A6B2}
    \definecolor{ansi-default-inverse-fg}{HTML}{FFFFFF}
    \definecolor{ansi-default-inverse-bg}{HTML}{000000}

    % common color for the border for error outputs.
    \definecolor{outerrorbackground}{HTML}{FFDFDF}

    % commands and environments needed by pandoc snippets
    % extracted from the output of `pandoc -s`
    \providecommand{\tightlist}{%
      \setlength{\itemsep}{0pt}\setlength{\parskip}{0pt}}
    \DefineVerbatimEnvironment{Highlighting}{Verbatim}{commandchars=\\\{\}}
    % Add ',fontsize=\small' for more characters per line
    \newenvironment{Shaded}{}{}
    \newcommand{\KeywordTok}[1]{\textcolor[rgb]{0.00,0.44,0.13}{\textbf{{#1}}}}
    \newcommand{\DataTypeTok}[1]{\textcolor[rgb]{0.56,0.13,0.00}{{#1}}}
    \newcommand{\DecValTok}[1]{\textcolor[rgb]{0.25,0.63,0.44}{{#1}}}
    \newcommand{\BaseNTok}[1]{\textcolor[rgb]{0.25,0.63,0.44}{{#1}}}
    \newcommand{\FloatTok}[1]{\textcolor[rgb]{0.25,0.63,0.44}{{#1}}}
    \newcommand{\CharTok}[1]{\textcolor[rgb]{0.25,0.44,0.63}{{#1}}}
    \newcommand{\StringTok}[1]{\textcolor[rgb]{0.25,0.44,0.63}{{#1}}}
    \newcommand{\CommentTok}[1]{\textcolor[rgb]{0.38,0.63,0.69}{\textit{{#1}}}}
    \newcommand{\OtherTok}[1]{\textcolor[rgb]{0.00,0.44,0.13}{{#1}}}
    \newcommand{\AlertTok}[1]{\textcolor[rgb]{1.00,0.00,0.00}{\textbf{{#1}}}}
    \newcommand{\FunctionTok}[1]{\textcolor[rgb]{0.02,0.16,0.49}{{#1}}}
    \newcommand{\RegionMarkerTok}[1]{{#1}}
    \newcommand{\ErrorTok}[1]{\textcolor[rgb]{1.00,0.00,0.00}{\textbf{{#1}}}}
    \newcommand{\NormalTok}[1]{{#1}}

    % Additional commands for more recent versions of Pandoc
    \newcommand{\ConstantTok}[1]{\textcolor[rgb]{0.53,0.00,0.00}{{#1}}}
    \newcommand{\SpecialCharTok}[1]{\textcolor[rgb]{0.25,0.44,0.63}{{#1}}}
    \newcommand{\VerbatimStringTok}[1]{\textcolor[rgb]{0.25,0.44,0.63}{{#1}}}
    \newcommand{\SpecialStringTok}[1]{\textcolor[rgb]{0.73,0.40,0.53}{{#1}}}
    \newcommand{\ImportTok}[1]{{#1}}
    \newcommand{\DocumentationTok}[1]{\textcolor[rgb]{0.73,0.13,0.13}{\textit{{#1}}}}
    \newcommand{\AnnotationTok}[1]{\textcolor[rgb]{0.38,0.63,0.69}{\textbf{\textit{{#1}}}}}
    \newcommand{\CommentVarTok}[1]{\textcolor[rgb]{0.38,0.63,0.69}{\textbf{\textit{{#1}}}}}
    \newcommand{\VariableTok}[1]{\textcolor[rgb]{0.10,0.09,0.49}{{#1}}}
    \newcommand{\ControlFlowTok}[1]{\textcolor[rgb]{0.00,0.44,0.13}{\textbf{{#1}}}}
    \newcommand{\OperatorTok}[1]{\textcolor[rgb]{0.40,0.40,0.40}{{#1}}}
    \newcommand{\BuiltInTok}[1]{{#1}}
    \newcommand{\ExtensionTok}[1]{{#1}}
    \newcommand{\PreprocessorTok}[1]{\textcolor[rgb]{0.74,0.48,0.00}{{#1}}}
    \newcommand{\AttributeTok}[1]{\textcolor[rgb]{0.49,0.56,0.16}{{#1}}}
    \newcommand{\InformationTok}[1]{\textcolor[rgb]{0.38,0.63,0.69}{\textbf{\textit{{#1}}}}}
    \newcommand{\WarningTok}[1]{\textcolor[rgb]{0.38,0.63,0.69}{\textbf{\textit{{#1}}}}}


    % Define a nice break command that doesn't care if a line doesn't already
    % exist.
    \def\br{\hspace*{\fill} \\* }
    % Math Jax compatibility definitions
    \def\gt{>}
    \def\lt{<}
    \let\Oldtex\TeX
    \let\Oldlatex\LaTeX
    \renewcommand{\TeX}{\textrm{\Oldtex}}
    \renewcommand{\LaTeX}{\textrm{\Oldlatex}}
    % Document parameters
    % Document title
    \title{REPORT}
    
    
    
    
    
    
    
% Pygments definitions
\makeatletter
\def\PY@reset{\let\PY@it=\relax \let\PY@bf=\relax%
    \let\PY@ul=\relax \let\PY@tc=\relax%
    \let\PY@bc=\relax \let\PY@ff=\relax}
\def\PY@tok#1{\csname PY@tok@#1\endcsname}
\def\PY@toks#1+{\ifx\relax#1\empty\else%
    \PY@tok{#1}\expandafter\PY@toks\fi}
\def\PY@do#1{\PY@bc{\PY@tc{\PY@ul{%
    \PY@it{\PY@bf{\PY@ff{#1}}}}}}}
\def\PY#1#2{\PY@reset\PY@toks#1+\relax+\PY@do{#2}}

\@namedef{PY@tok@w}{\def\PY@tc##1{\textcolor[rgb]{0.73,0.73,0.73}{##1}}}
\@namedef{PY@tok@c}{\let\PY@it=\textit\def\PY@tc##1{\textcolor[rgb]{0.24,0.48,0.48}{##1}}}
\@namedef{PY@tok@cp}{\def\PY@tc##1{\textcolor[rgb]{0.61,0.40,0.00}{##1}}}
\@namedef{PY@tok@k}{\let\PY@bf=\textbf\def\PY@tc##1{\textcolor[rgb]{0.00,0.50,0.00}{##1}}}
\@namedef{PY@tok@kp}{\def\PY@tc##1{\textcolor[rgb]{0.00,0.50,0.00}{##1}}}
\@namedef{PY@tok@kt}{\def\PY@tc##1{\textcolor[rgb]{0.69,0.00,0.25}{##1}}}
\@namedef{PY@tok@o}{\def\PY@tc##1{\textcolor[rgb]{0.40,0.40,0.40}{##1}}}
\@namedef{PY@tok@ow}{\let\PY@bf=\textbf\def\PY@tc##1{\textcolor[rgb]{0.67,0.13,1.00}{##1}}}
\@namedef{PY@tok@nb}{\def\PY@tc##1{\textcolor[rgb]{0.00,0.50,0.00}{##1}}}
\@namedef{PY@tok@nf}{\def\PY@tc##1{\textcolor[rgb]{0.00,0.00,1.00}{##1}}}
\@namedef{PY@tok@nc}{\let\PY@bf=\textbf\def\PY@tc##1{\textcolor[rgb]{0.00,0.00,1.00}{##1}}}
\@namedef{PY@tok@nn}{\let\PY@bf=\textbf\def\PY@tc##1{\textcolor[rgb]{0.00,0.00,1.00}{##1}}}
\@namedef{PY@tok@ne}{\let\PY@bf=\textbf\def\PY@tc##1{\textcolor[rgb]{0.80,0.25,0.22}{##1}}}
\@namedef{PY@tok@nv}{\def\PY@tc##1{\textcolor[rgb]{0.10,0.09,0.49}{##1}}}
\@namedef{PY@tok@no}{\def\PY@tc##1{\textcolor[rgb]{0.53,0.00,0.00}{##1}}}
\@namedef{PY@tok@nl}{\def\PY@tc##1{\textcolor[rgb]{0.46,0.46,0.00}{##1}}}
\@namedef{PY@tok@ni}{\let\PY@bf=\textbf\def\PY@tc##1{\textcolor[rgb]{0.44,0.44,0.44}{##1}}}
\@namedef{PY@tok@na}{\def\PY@tc##1{\textcolor[rgb]{0.41,0.47,0.13}{##1}}}
\@namedef{PY@tok@nt}{\let\PY@bf=\textbf\def\PY@tc##1{\textcolor[rgb]{0.00,0.50,0.00}{##1}}}
\@namedef{PY@tok@nd}{\def\PY@tc##1{\textcolor[rgb]{0.67,0.13,1.00}{##1}}}
\@namedef{PY@tok@s}{\def\PY@tc##1{\textcolor[rgb]{0.73,0.13,0.13}{##1}}}
\@namedef{PY@tok@sd}{\let\PY@it=\textit\def\PY@tc##1{\textcolor[rgb]{0.73,0.13,0.13}{##1}}}
\@namedef{PY@tok@si}{\let\PY@bf=\textbf\def\PY@tc##1{\textcolor[rgb]{0.64,0.35,0.47}{##1}}}
\@namedef{PY@tok@se}{\let\PY@bf=\textbf\def\PY@tc##1{\textcolor[rgb]{0.67,0.36,0.12}{##1}}}
\@namedef{PY@tok@sr}{\def\PY@tc##1{\textcolor[rgb]{0.64,0.35,0.47}{##1}}}
\@namedef{PY@tok@ss}{\def\PY@tc##1{\textcolor[rgb]{0.10,0.09,0.49}{##1}}}
\@namedef{PY@tok@sx}{\def\PY@tc##1{\textcolor[rgb]{0.00,0.50,0.00}{##1}}}
\@namedef{PY@tok@m}{\def\PY@tc##1{\textcolor[rgb]{0.40,0.40,0.40}{##1}}}
\@namedef{PY@tok@gh}{\let\PY@bf=\textbf\def\PY@tc##1{\textcolor[rgb]{0.00,0.00,0.50}{##1}}}
\@namedef{PY@tok@gu}{\let\PY@bf=\textbf\def\PY@tc##1{\textcolor[rgb]{0.50,0.00,0.50}{##1}}}
\@namedef{PY@tok@gd}{\def\PY@tc##1{\textcolor[rgb]{0.63,0.00,0.00}{##1}}}
\@namedef{PY@tok@gi}{\def\PY@tc##1{\textcolor[rgb]{0.00,0.52,0.00}{##1}}}
\@namedef{PY@tok@gr}{\def\PY@tc##1{\textcolor[rgb]{0.89,0.00,0.00}{##1}}}
\@namedef{PY@tok@ge}{\let\PY@it=\textit}
\@namedef{PY@tok@gs}{\let\PY@bf=\textbf}
\@namedef{PY@tok@ges}{\let\PY@bf=\textbf\let\PY@it=\textit}
\@namedef{PY@tok@gp}{\let\PY@bf=\textbf\def\PY@tc##1{\textcolor[rgb]{0.00,0.00,0.50}{##1}}}
\@namedef{PY@tok@go}{\def\PY@tc##1{\textcolor[rgb]{0.44,0.44,0.44}{##1}}}
\@namedef{PY@tok@gt}{\def\PY@tc##1{\textcolor[rgb]{0.00,0.27,0.87}{##1}}}
\@namedef{PY@tok@err}{\def\PY@bc##1{{\setlength{\fboxsep}{\string -\fboxrule}\fcolorbox[rgb]{1.00,0.00,0.00}{1,1,1}{\strut ##1}}}}
\@namedef{PY@tok@kc}{\let\PY@bf=\textbf\def\PY@tc##1{\textcolor[rgb]{0.00,0.50,0.00}{##1}}}
\@namedef{PY@tok@kd}{\let\PY@bf=\textbf\def\PY@tc##1{\textcolor[rgb]{0.00,0.50,0.00}{##1}}}
\@namedef{PY@tok@kn}{\let\PY@bf=\textbf\def\PY@tc##1{\textcolor[rgb]{0.00,0.50,0.00}{##1}}}
\@namedef{PY@tok@kr}{\let\PY@bf=\textbf\def\PY@tc##1{\textcolor[rgb]{0.00,0.50,0.00}{##1}}}
\@namedef{PY@tok@bp}{\def\PY@tc##1{\textcolor[rgb]{0.00,0.50,0.00}{##1}}}
\@namedef{PY@tok@fm}{\def\PY@tc##1{\textcolor[rgb]{0.00,0.00,1.00}{##1}}}
\@namedef{PY@tok@vc}{\def\PY@tc##1{\textcolor[rgb]{0.10,0.09,0.49}{##1}}}
\@namedef{PY@tok@vg}{\def\PY@tc##1{\textcolor[rgb]{0.10,0.09,0.49}{##1}}}
\@namedef{PY@tok@vi}{\def\PY@tc##1{\textcolor[rgb]{0.10,0.09,0.49}{##1}}}
\@namedef{PY@tok@vm}{\def\PY@tc##1{\textcolor[rgb]{0.10,0.09,0.49}{##1}}}
\@namedef{PY@tok@sa}{\def\PY@tc##1{\textcolor[rgb]{0.73,0.13,0.13}{##1}}}
\@namedef{PY@tok@sb}{\def\PY@tc##1{\textcolor[rgb]{0.73,0.13,0.13}{##1}}}
\@namedef{PY@tok@sc}{\def\PY@tc##1{\textcolor[rgb]{0.73,0.13,0.13}{##1}}}
\@namedef{PY@tok@dl}{\def\PY@tc##1{\textcolor[rgb]{0.73,0.13,0.13}{##1}}}
\@namedef{PY@tok@s2}{\def\PY@tc##1{\textcolor[rgb]{0.73,0.13,0.13}{##1}}}
\@namedef{PY@tok@sh}{\def\PY@tc##1{\textcolor[rgb]{0.73,0.13,0.13}{##1}}}
\@namedef{PY@tok@s1}{\def\PY@tc##1{\textcolor[rgb]{0.73,0.13,0.13}{##1}}}
\@namedef{PY@tok@mb}{\def\PY@tc##1{\textcolor[rgb]{0.40,0.40,0.40}{##1}}}
\@namedef{PY@tok@mf}{\def\PY@tc##1{\textcolor[rgb]{0.40,0.40,0.40}{##1}}}
\@namedef{PY@tok@mh}{\def\PY@tc##1{\textcolor[rgb]{0.40,0.40,0.40}{##1}}}
\@namedef{PY@tok@mi}{\def\PY@tc##1{\textcolor[rgb]{0.40,0.40,0.40}{##1}}}
\@namedef{PY@tok@il}{\def\PY@tc##1{\textcolor[rgb]{0.40,0.40,0.40}{##1}}}
\@namedef{PY@tok@mo}{\def\PY@tc##1{\textcolor[rgb]{0.40,0.40,0.40}{##1}}}
\@namedef{PY@tok@ch}{\let\PY@it=\textit\def\PY@tc##1{\textcolor[rgb]{0.24,0.48,0.48}{##1}}}
\@namedef{PY@tok@cm}{\let\PY@it=\textit\def\PY@tc##1{\textcolor[rgb]{0.24,0.48,0.48}{##1}}}
\@namedef{PY@tok@cpf}{\let\PY@it=\textit\def\PY@tc##1{\textcolor[rgb]{0.24,0.48,0.48}{##1}}}
\@namedef{PY@tok@c1}{\let\PY@it=\textit\def\PY@tc##1{\textcolor[rgb]{0.24,0.48,0.48}{##1}}}
\@namedef{PY@tok@cs}{\let\PY@it=\textit\def\PY@tc##1{\textcolor[rgb]{0.24,0.48,0.48}{##1}}}

\def\PYZbs{\char`\\}
\def\PYZus{\char`\_}
\def\PYZob{\char`\{}
\def\PYZcb{\char`\}}
\def\PYZca{\char`\^}
\def\PYZam{\char`\&}
\def\PYZlt{\char`\<}
\def\PYZgt{\char`\>}
\def\PYZsh{\char`\#}
\def\PYZpc{\char`\%}
\def\PYZdl{\char`\$}
\def\PYZhy{\char`\-}
\def\PYZsq{\char`\'}
\def\PYZdq{\char`\"}
\def\PYZti{\char`\~}
% for compatibility with earlier versions
\def\PYZat{@}
\def\PYZlb{[}
\def\PYZrb{]}
\makeatother


    % For linebreaks inside Verbatim environment from package fancyvrb.
    \makeatletter
        \newbox\Wrappedcontinuationbox
        \newbox\Wrappedvisiblespacebox
        \newcommand*\Wrappedvisiblespace {\textcolor{red}{\textvisiblespace}}
        \newcommand*\Wrappedcontinuationsymbol {\textcolor{red}{\llap{\tiny$\m@th\hookrightarrow$}}}
        \newcommand*\Wrappedcontinuationindent {3ex }
        \newcommand*\Wrappedafterbreak {\kern\Wrappedcontinuationindent\copy\Wrappedcontinuationbox}
        % Take advantage of the already applied Pygments mark-up to insert
        % potential linebreaks for TeX processing.
        %        {, <, #, %, $, ' and ": go to next line.
        %        _, }, ^, &, >, - and ~: stay at end of broken line.
        % Use of \textquotesingle for straight quote.
        \newcommand*\Wrappedbreaksatspecials {%
            \def\PYGZus{\discretionary{\char`\_}{\Wrappedafterbreak}{\char`\_}}%
            \def\PYGZob{\discretionary{}{\Wrappedafterbreak\char`\{}{\char`\{}}%
            \def\PYGZcb{\discretionary{\char`\}}{\Wrappedafterbreak}{\char`\}}}%
            \def\PYGZca{\discretionary{\char`\^}{\Wrappedafterbreak}{\char`\^}}%
            \def\PYGZam{\discretionary{\char`\&}{\Wrappedafterbreak}{\char`\&}}%
            \def\PYGZlt{\discretionary{}{\Wrappedafterbreak\char`\<}{\char`\<}}%
            \def\PYGZgt{\discretionary{\char`\>}{\Wrappedafterbreak}{\char`\>}}%
            \def\PYGZsh{\discretionary{}{\Wrappedafterbreak\char`\#}{\char`\#}}%
            \def\PYGZpc{\discretionary{}{\Wrappedafterbreak\char`\%}{\char`\%}}%
            \def\PYGZdl{\discretionary{}{\Wrappedafterbreak\char`\$}{\char`\$}}%
            \def\PYGZhy{\discretionary{\char`\-}{\Wrappedafterbreak}{\char`\-}}%
            \def\PYGZsq{\discretionary{}{\Wrappedafterbreak\textquotesingle}{\textquotesingle}}%
            \def\PYGZdq{\discretionary{}{\Wrappedafterbreak\char`\"}{\char`\"}}%
            \def\PYGZti{\discretionary{\char`\~}{\Wrappedafterbreak}{\char`\~}}%
        }
        % Some characters . , ; ? ! / are not pygmentized.
        % This macro makes them "active" and they will insert potential linebreaks
        \newcommand*\Wrappedbreaksatpunct {%
            \lccode`\~`\.\lowercase{\def~}{\discretionary{\hbox{\char`\.}}{\Wrappedafterbreak}{\hbox{\char`\.}}}%
            \lccode`\~`\,\lowercase{\def~}{\discretionary{\hbox{\char`\,}}{\Wrappedafterbreak}{\hbox{\char`\,}}}%
            \lccode`\~`\;\lowercase{\def~}{\discretionary{\hbox{\char`\;}}{\Wrappedafterbreak}{\hbox{\char`\;}}}%
            \lccode`\~`\:\lowercase{\def~}{\discretionary{\hbox{\char`\:}}{\Wrappedafterbreak}{\hbox{\char`\:}}}%
            \lccode`\~`\?\lowercase{\def~}{\discretionary{\hbox{\char`\?}}{\Wrappedafterbreak}{\hbox{\char`\?}}}%
            \lccode`\~`\!\lowercase{\def~}{\discretionary{\hbox{\char`\!}}{\Wrappedafterbreak}{\hbox{\char`\!}}}%
            \lccode`\~`\/\lowercase{\def~}{\discretionary{\hbox{\char`\/}}{\Wrappedafterbreak}{\hbox{\char`\/}}}%
            \catcode`\.\active
            \catcode`\,\active
            \catcode`\;\active
            \catcode`\:\active
            \catcode`\?\active
            \catcode`\!\active
            \catcode`\/\active
            \lccode`\~`\~
        }
    \makeatother

    \let\OriginalVerbatim=\Verbatim
    \makeatletter
    \renewcommand{\Verbatim}[1][1]{%
        %\parskip\z@skip
        \sbox\Wrappedcontinuationbox {\Wrappedcontinuationsymbol}%
        \sbox\Wrappedvisiblespacebox {\FV@SetupFont\Wrappedvisiblespace}%
        \def\FancyVerbFormatLine ##1{\hsize\linewidth
            \vtop{\raggedright\hyphenpenalty\z@\exhyphenpenalty\z@
                \doublehyphendemerits\z@\finalhyphendemerits\z@
                \strut ##1\strut}%
        }%
        % If the linebreak is at a space, the latter will be displayed as visible
        % space at end of first line, and a continuation symbol starts next line.
        % Stretch/shrink are however usually zero for typewriter font.
        \def\FV@Space {%
            \nobreak\hskip\z@ plus\fontdimen3\font minus\fontdimen4\font
            \discretionary{\copy\Wrappedvisiblespacebox}{\Wrappedafterbreak}
            {\kern\fontdimen2\font}%
        }%

        % Allow breaks at special characters using \PYG... macros.
        \Wrappedbreaksatspecials
        % Breaks at punctuation characters . , ; ? ! and / need catcode=\active
        \OriginalVerbatim[#1,codes*=\Wrappedbreaksatpunct]%
    }
    \makeatother

    % Exact colors from NB
    \definecolor{incolor}{HTML}{303F9F}
    \definecolor{outcolor}{HTML}{D84315}
    \definecolor{cellborder}{HTML}{CFCFCF}
    \definecolor{cellbackground}{HTML}{F7F7F7}

    % prompt
    \makeatletter
    \newcommand{\boxspacing}{\kern\kvtcb@left@rule\kern\kvtcb@boxsep}
    \makeatother
    \newcommand{\prompt}[4]{
        {\ttfamily\llap{{\color{#2}[#3]:\hspace{3pt}#4}}\vspace{-\baselineskip}}
    }
    

    
    % Prevent overflowing lines due to hard-to-break entities
    \sloppy
    % Setup hyperref package
    \hypersetup{
      breaklinks=true,  % so long urls are correctly broken across lines
      colorlinks=true,
      urlcolor=urlcolor,
      linkcolor=linkcolor,
      citecolor=citecolor,
      }
    % Slightly bigger margins than the latex defaults
    
    \geometry{verbose,tmargin=1in,bmargin=1in,lmargin=1in,rmargin=1in}
    
    

\begin{document}
    
    \maketitle
    
    

    
    \(\Huge{\textbf{Clouds Course}}\)

\(\small{\textbf{Assignment 1 Deliverable}}\)

    \begin{center}\rule{0.5\linewidth}{0.5pt}\end{center}

\(\color{blue}\textbf{Name :}\)

\(\boxed{\text{        Brice Robert          }}\)

\(\color{blue}\textbf{Email :}\)

\(\boxed{\text{robert@eurecom.fr}}\)

\begin{center}\rule{0.5\linewidth}{0.5pt}\end{center}

    \section{Create a free Azure account}\label{create-a-free-azure-account}

    \(\color{blue}\textbf{Did you run into any problems with creating an Azure account?}\)

\begin{center}\rule{0.5\linewidth}{0.5pt}\end{center}

\(\color{blue} \textbf{Your answer:}\) Having to deal with the personal
account before hand was confusing but finally the Azure Account was
credited using the EURECOM's email address.

    \section{Deploy a website on Virtual
Machine}\label{deploy-a-website-on-virtual-machine}

    \(\color{blue}\textbf{What is the VM instance type you used? How many cores and memory does it have?}\)

\begin{center}\rule{0.5\linewidth}{0.5pt}\end{center}

\(\color{blue} \textbf{Your answer:}\)

\begin{Shaded}
\begin{Highlighting}[]
\VariableTok{VM\_SIZE}\OperatorTok{=}\VariableTok{$(}\ExtensionTok{az}\NormalTok{ vm show }\AttributeTok{{-}{-}name}\NormalTok{ lab1{-}vm }\DataTypeTok{\textbackslash{}}
    \AttributeTok{{-}{-}resource{-}group}\NormalTok{ lab1{-}resources }\AttributeTok{{-}{-}query} \StringTok{"hardwareProfile.vmSize"} \AttributeTok{{-}o}\NormalTok{ tsv}\VariableTok{)}
\ExtensionTok{az}\NormalTok{ vm list{-}sizes }\AttributeTok{{-}{-}location}\NormalTok{ east{-}us }\AttributeTok{{-}{-}query} \StringTok{"[?name==\textquotesingle{}}\VariableTok{$VM\_SIZE}\StringTok{\textquotesingle{}]"} \AttributeTok{{-}o}\NormalTok{ table}
\end{Highlighting}
\end{Shaded}

\begin{quote}
Returns
\end{quote}

\begin{Shaded}
\begin{Highlighting}[]
\NormalTok{MaxDataDiskCount  MemoryInMB  Name           NumberOfCores  OsDiskSizeInMB  ResourceDiskSizeInMB}
\OperatorTok{{-}{-}{-}{-}{-}{-}{-}{-}{-}{-}{-}{-}{-}{-}{-}{-}}  \OperatorTok{{-}{-}{-}{-}{-}{-}{-}{-}{-}{-}}  \OperatorTok{{-}{-}{-}{-}{-}{-}{-}{-}{-}{-}{-}{-}{-}}  \OperatorTok{{-}{-}{-}{-}{-}{-}{-}{-}{-}{-}{-}{-}{-}}  \OperatorTok{{-}{-}{-}{-}{-}{-}{-}{-}{-}{-}{-}{-}{-}{-}}  \OperatorTok{{-}{-}{-}{-}{-}{-}{-}{-}{-}{-}{-}{-}{-}{-}{-}{-}{-}{-}{-}{-}{-}{-}}
\DecValTok{2}                 \DecValTok{512}\NormalTok{         Standard\_B1ls  }\DecValTok{1}              \DecValTok{1047552}         \DecValTok{4096}
\end{Highlighting}
\end{Shaded}

\begin{itemize}
\tightlist
\item
  Name: Standard\_B1ls (The VM instance type).
\item
  NumberOfCores: 1 (This instance has 1 core).
\item
  MemoryInMB: 512 MB (The VM has 512 MB of RAM).
\end{itemize}

\begin{center}\rule{0.5\linewidth}{0.5pt}\end{center}

\(\color{blue}\textbf{What is the size of the virtual disk used by the VM? What type is it?}\)

\begin{center}\rule{0.5\linewidth}{0.5pt}\end{center}

\(\color{blue} \textbf{Your answer:}\)

\begin{Shaded}
\begin{Highlighting}[]
\ExtensionTok{az}\NormalTok{ vm show }\AttributeTok{{-}{-}name}\NormalTok{ lab1{-}vm }\AttributeTok{{-}{-}resource{-}group}\NormalTok{ lab1{-}resources }\DataTypeTok{\textbackslash{}}
\NormalTok{{-}{-}query }\StringTok{"storageProfile.osDisk.\{Size:diskSizeGb, Type:managedDisk.storageAccountType\}"} \AttributeTok{{-}o}\NormalTok{ table}
\end{Highlighting}
\end{Shaded}

\begin{quote}
Returns
\end{quote}

\begin{Shaded}
\begin{Highlighting}[]
\NormalTok{Size    }\FunctionTok{Type}
\OperatorTok{{-}{-}{-}{-}{-}{-}}  \OperatorTok{{-}{-}{-}{-}{-}{-}{-}{-}{-}{-}{-}{-}}
\DecValTok{30}\NormalTok{      Standard\_LRS}
\end{Highlighting}
\end{Shaded}

\begin{itemize}
\tightlist
\item
  Size: The virtual disk size is defined by the diskSizeGb field (e.g.,
  30 GB).
\item
  Type: The storage type is defined by the storageAccountType field
  (e.g., Standard\_LRS).
\end{itemize}

\begin{center}\rule{0.5\linewidth}{0.5pt}\end{center}

\(\color{blue}\textbf{What is the public ip address of the VM where we can access your site?}\)

\begin{center}\rule{0.5\linewidth}{0.5pt}\end{center}

\(\color{blue} \textbf{Your answer:}\)

\begin{verbatim}
az vm list-ip-addresses --resource-group lab1-resources --output table
\end{verbatim}

\begin{quote}
Returns
\end{quote}

\begin{Shaded}
\begin{Highlighting}[]
\NormalTok{VirtualMachine    PublicIPAddresses    PrivateIPAddresses}
\OperatorTok{{-}{-}{-}{-}{-}{-}{-}{-}{-}{-}{-}{-}{-}{-}{-}{-}}  \OperatorTok{{-}{-}{-}{-}{-}{-}{-}{-}{-}{-}{-}{-}{-}{-}{-}{-}{-}{-}{-}}  \OperatorTok{{-}{-}{-}{-}{-}{-}{-}{-}{-}{-}{-}{-}{-}{-}{-}{-}{-}{-}{-}{-}}
\NormalTok{lab1{-}vm           }\DecValTok{52.168}\OperatorTok{.}\DecValTok{95.147}        \DecValTok{10.0}\OperatorTok{.}\DecValTok{1.4}
\end{Highlighting}
\end{Shaded}

\paragraph{Public IP Address:}\label{public-ip-address}

http://52.168.95.147

    \section{Provision disks in VM and measure
IOPS}\label{provision-disks-in-vm-and-measure-iops}

    \(\color{blue}\textbf{What VM type did you resize to?}\)

\(\color{blue}\textbf{What is the difference in spec between this type and the one you used before?}\)

\begin{center}\rule{0.5\linewidth}{0.5pt}\end{center}

\(\color{blue} \textbf{Your answer:}\)

\begin{verbatim}
VM Type: Standard_DS3_v2
\end{verbatim}

\begin{longtable}[]{@{}
  >{\raggedright\arraybackslash}p{(\linewidth - 4\tabcolsep) * \real{0.3333}}
  >{\raggedright\arraybackslash}p{(\linewidth - 4\tabcolsep) * \real{0.3333}}
  >{\raggedright\arraybackslash}p{(\linewidth - 4\tabcolsep) * \real{0.3333}}@{}}
\toprule\noalign{}
\begin{minipage}[b]{\linewidth}\raggedright
\textbf{Specification}
\end{minipage} & \begin{minipage}[b]{\linewidth}\raggedright
\textbf{Standard\_DS3\_v2}
\end{minipage} & \begin{minipage}[b]{\linewidth}\raggedright
\textbf{Standard\_B1ls}
\end{minipage} \\
\midrule\noalign{}
\endhead
\bottomrule\noalign{}
\endlastfoot
\textbf{vCPUs} & 4 & 1 \\
\textbf{Memory} & 14 GiB & 0.5 GiB \\
\textbf{Temp Storage (SSD)} & 28 GiB & 4 GiB \\
\textbf{Max Data Disks} & 16 & 2 \\
\textbf{Max IOPS (Cached/Uncached)} & 16,000 / 12,800 & 200 / 320 \\
\textbf{Max Network Interfaces} & 4 & 2 \\
\textbf{Accelerated Networking} & Supported & Not Supported \\
\textbf{Use Case} & Enterprise apps, DBs & Low-utilization, lightweight
apps \\
\end{longtable}

\textbf{Summary}:\\
The \textbf{DS3\_v2} is a high-performance VM for demanding workloads,
while the \textbf{B1ls} is a budget-friendly option for lightweight
tasks.

\begin{center}\rule{0.5\linewidth}{0.5pt}\end{center}

\(\color{blue}\textbf{What type of disks did you add? What were their sizes?}\)

\(\color{blue}\textbf{According to Azure, what is their expected sequencing bandwidth and random IOPS?}\)

\begin{center}\rule{0.5\linewidth}{0.5pt}\end{center}

\(\color{blue} \textbf{Your answer:}\)

Type of disks and sizes:

\begin{itemize}
\tightlist
\item
  \textbf{Standard SSD}: 50 GiB, \texttt{/dev/sdb}
  (\texttt{StandardSSD\_LRS})\\
\item
  \textbf{Premium SSD}: 50 GiB, \texttt{/dev/sdc}
  (\texttt{Premium\_LRS})
\end{itemize}

The virtual machine `lab1-vm' has two attached data disks:

\begin{longtable}[]{@{}
  >{\raggedright\arraybackslash}p{(\linewidth - 12\tabcolsep) * \real{0.0568}}
  >{\raggedright\arraybackslash}p{(\linewidth - 12\tabcolsep) * \real{0.2273}}
  >{\raggedright\arraybackslash}p{(\linewidth - 12\tabcolsep) * \real{0.1023}}
  >{\raggedright\arraybackslash}p{(\linewidth - 12\tabcolsep) * \real{0.1591}}
  >{\raggedright\arraybackslash}p{(\linewidth - 12\tabcolsep) * \real{0.1364}}
  >{\raggedright\arraybackslash}p{(\linewidth - 12\tabcolsep) * \real{0.1591}}
  >{\raggedright\arraybackslash}p{(\linewidth - 12\tabcolsep) * \real{0.1591}}@{}}
\toprule\noalign{}
\begin{minipage}[b]{\linewidth}\raggedright
Lun
\end{minipage} & \begin{minipage}[b]{\linewidth}\raggedright
Name
\end{minipage} & \begin{minipage}[b]{\linewidth}\raggedright
Caching
\end{minipage} & \begin{minipage}[b]{\linewidth}\raggedright
CreateOption
\end{minipage} & \begin{minipage}[b]{\linewidth}\raggedright
DiskSizeGb
\end{minipage} & \begin{minipage}[b]{\linewidth}\raggedright
ToBeDetached
\end{minipage} & \begin{minipage}[b]{\linewidth}\raggedright
DeleteOption
\end{minipage} \\
\midrule\noalign{}
\endhead
\bottomrule\noalign{}
\endlastfoot
0 & standard-ssd-disk & ReadOnly & Attach & 50 & False & Detach \\
1 & premium-ssd-disk & ReadOnly & Attach & 50 & False & Detach \\
\end{longtable}

Where:

\begin{itemize}
\tightlist
\item
  \texttt{/dev/sdb} corresponds to `standard-ssd-disk' (Standard SSD).
\item
  \texttt{/dev/sdc} corresponds to `premium-ssd-disk' (Premium SSD).
\end{itemize}

According to Azure's documentation, the expected performance metrics for
these disk types are:

\textbf{Standard SSD (E10 - 50 GiB):} - \textbf{IOPS:} 500 -
\textbf{Throughput:} 60 MB/s

\textbf{Premium SSD (P10 - 50 GiB):} - \textbf{IOPS:} 500 -
\textbf{Throughput:} 100 MB/s

These specifications are detailed in Azure's managed disk pricing
documentation.

Comparing these expected values to the \texttt{fio} test results:

\begin{itemize}
\tightlist
\item
  \textbf{\texttt{/dev/sdb} (Standard SSD):}

  \begin{itemize}
  \tightlist
  \item
    Measured IOPS: 3,128
  \item
    Measured Throughput: 196 MiB/s (approximately 205 MB/s)
  \end{itemize}
\item
  \textbf{\texttt{/dev/sdc} (Premium SSD):}

  \begin{itemize}
  \tightlist
  \item
    Measured IOPS: 421
  \item
    Measured Throughput: 26.3 MiB/s (approximately 27.6 MB/s)
  \end{itemize}
\end{itemize}

The measured performance for \texttt{/dev/sdb} significantly exceeds
Azure's documented expectations for a Standard SSD of this size, while
\texttt{/dev/sdc} falls below the expected performance for a Premium
SSD.

I checked if there was a mismatch between the 2 disks but couldn't find
any.

\begin{center}\rule{0.5\linewidth}{0.5pt}\end{center}

\(\color{blue}\textbf{What is the sequencing bandwidth and random IOPS you measured with FIO?}\)

\begin{center}\rule{0.5\linewidth}{0.5pt}\end{center}

\(\color{blue} \textbf{Your answer:}\)

\begin{itemize}
\tightlist
\item
  \textbf{Standard SSD (\texttt{/dev/sdb})}:

  \begin{itemize}
  \tightlist
  \item
    \textbf{Sequencing Bandwidth}: 196 MiB/s\\
  \item
    \textbf{Random IOPS}: 3,128
  \end{itemize}
\item
  \textbf{Premium SSD (\texttt{/dev/sdc})}:

  \begin{itemize}
  \tightlist
  \item
    \textbf{Sequencing Bandwidth}: 26.3 MiB/s\\
  \item
    \textbf{Random IOPS}: 421
  \end{itemize}
\end{itemize}

Here's a detailed explanation of the \textbf{FIO test results} for the
two disks:

\subsubsection{\texorpdfstring{\textbf{Standard SSD
(\texttt{/dev/sdb}):}}{Standard SSD (/dev/sdb):}}\label{standard-ssd-devsdb}

\begin{enumerate}
\def\labelenumi{\arabic{enumi}.}
\tightlist
\item
  \textbf{Sequencing Bandwidth (196 MiB/s):}

  \begin{itemize}
  \tightlist
  \item
    This indicates the disk's performance when reading/writing large
    blocks of data sequentially.
  \item
    The measured value of \textbf{196 MiB/s} is typical for a Standard
    SSD, as sequential operations leverage the SSD's design for high
    throughput.
  \end{itemize}
\item
  \textbf{Random IOPS (3,128):}

  \begin{itemize}
  \tightlist
  \item
    \textbf{IOPS (Input/Output Operations Per Second)} measures the
    performance of small, random read/write operations (4 KB blocks).
  \item
    A higher IOPS value reflects the disk's ability to handle small,
    frequent operations efficiently.
  \item
    The value of \textbf{3,128 IOPS} aligns with the Standard SSD's
    performance for workloads with random access patterns (e.g.,
    database queries).
  \end{itemize}
\end{enumerate}

\begin{center}\rule{0.5\linewidth}{0.5pt}\end{center}

\subsubsection{\texorpdfstring{\textbf{Premium SSD
(\texttt{/dev/sdc}):}}{Premium SSD (/dev/sdc):}}\label{premium-ssd-devsdc}

\begin{enumerate}
\def\labelenumi{\arabic{enumi}.}
\tightlist
\item
  \textbf{Sequencing Bandwidth (26.3 MiB/s):}

  \begin{itemize}
  \tightlist
  \item
    Surprisingly lower than expected for a Premium SSD. This might
    indicate:

    \begin{itemize}
    \tightlist
    \item
      Misconfiguration in the disk or VM.
    \item
      Bottlenecks such as network bandwidth limits (if the disk is
      remotely managed).
    \end{itemize}
  \end{itemize}
\item
  \textbf{Random IOPS (421):}

  \begin{itemize}
  \tightlist
  \item
    The random IOPS measured is significantly below Azure's advertised
    range for Premium SSDs, which typically exceed 5,000 IOPS for this
    size. This low value may again reflect:

    \begin{itemize}
    \tightlist
    \item
      Incorrect caching or mounting settings.
    \item
      Throttling due to VM configuration or tier mismatch.
    \end{itemize}
  \end{itemize}
\end{enumerate}

\begin{center}\rule{0.5\linewidth}{0.5pt}\end{center}

\subsubsection{\texorpdfstring{\textbf{Possible Discrepancies to
Investigate:}}{Possible Discrepancies to Investigate:}}\label{possible-discrepancies-to-investigate}

\begin{itemize}
\tightlist
\item
  Confirm the disk types using Azure CLI or other tools
  (\texttt{az\ disk\ show}).
\item
  Ensure the disks are attached with proper configurations (e.g.,
  Premium SSDs benefit from \texttt{ReadWrite} caching).
\item
  Review the VM size to confirm it supports the performance expected for
  these disk types.
\item
  Rerun tests with adjusted parameters, such as \texttt{numjobs},
  \texttt{iodepth}, and block sizes (\texttt{bs}) to ensure realistic
  workloads.
\end{itemize}

\paragraph{All the FIO testing results can be seen
here:}\label{all-the-fio-testing-results-can-be-seen-here}

\begin{itemize}
\tightlist
\item
  https://github.com/setrar/Clouds/tree/main/Labs/Azure/lab1/IaC\#fio
\end{itemize}

    \section{Deploy dockerized static site via Azure
ACR}\label{deploy-dockerized-static-site-via-azure-acr}

    \(\color{blue}\textbf{What is the public ip address of the ACI }\)

\(\color{blue}\textbf{deployed container instance where we can access your site?}\)

\begin{center}\rule{0.5\linewidth}{0.5pt}\end{center}

\(\color{blue} \textbf{Your answer:}\)

http://acrclouds2025eurbr.eastus.azurecontainer.io

\paragraph{All the manual commands needed to manage the ACR can be found
here}\label{all-the-manual-commands-needed-to-manage-the-acr-can-be-found-here}

https://github.com/setrar/Clouds/blob/main/Labs/Azure/lab1/IaC/README.md\#docker

\begin{center}\rule{0.5\linewidth}{0.5pt}\end{center}

\(\color{blue}\textbf{What is the size of the container image used?}\)

\begin{center}\rule{0.5\linewidth}{0.5pt}\end{center}

\(\color{blue} \textbf{Your answer:}\)

\(\color{blue}\textbf{ (You can find this by listing the size of all images in your container registry)}\)

\begin{verbatim}
docker image ls
\end{verbatim}

\begin{quote}
Returns:
\end{quote}

\begin{Shaded}
\begin{Highlighting}[]
\NormalTok{REPOSITORY                                  TAG       IMAGE ID       CREATED             SIZE}
\NormalTok{acrclouds2025eurbr}\OperatorTok{.}\FunctionTok{azurecr}\OperatorTok{.}\FunctionTok{io}\OperatorTok{/}\NormalTok{static{-}site   v1        e62dc83210d6   About an hour ago   192MB}
\NormalTok{static{-}site                                 latest    e62dc83210d6   About an hour ago   192MB}
\NormalTok{nginx                                       latest    9bea9f2796e2   }\DecValTok{7}\NormalTok{ weeks ago         192MB}
\end{Highlighting}
\end{Shaded}

Docker Image size on the VM: 192MB

\begin{verbatim}
az acr repository show-manifests --name acrclouds2025eurbr \
   --repository static-site --detail --query '[].{Image: tags[0], Size: imageSize}' --output table
\end{verbatim}

\begin{quote}
Returns
\end{quote}

\begin{Shaded}
\begin{Highlighting}[]
\NormalTok{This command has been deprecated and will be removed }\KeywordTok{in}\NormalTok{ a future release}\OperatorTok{.}\NormalTok{ Use }\StringTok{\textquotesingle{}acr manifest list{-}metadata\textquotesingle{}}\NormalTok{ instead}\OperatorTok{.}
\NormalTok{Image    Size}
\OperatorTok{{-}{-}{-}{-}{-}{-}{-}}  \OperatorTok{{-}{-}{-}{-}{-}{-}{-}{-}}
\NormalTok{v1       }\DecValTok{72972941}
\end{Highlighting}
\end{Shaded}

Container Size on ACR: 69MB

\(\text{Size in MB} = \frac{72972941}{1048576} \approx 69.58 \, \text{MB}\)

So, \(72972941 \, \text{bytes} \approx 69 \, \text{MB}\) (rounded to the
nearest whole number).

Which is \(\frac{1}{2}\) of the original docker image size

    \section{Create Azure webapp static
site}\label{create-azure-webapp-static-site}

    \(\color{blue}\textbf{What is the public address of the webapps static site?}\)

\begin{center}\rule{0.5\linewidth}{0.5pt}\end{center}

\(\color{blue} \textbf{Your answer:}\)

https://webappclouds2025eurbr.azurewebsites.net

\paragraph{The IaC file that created the WebApp can be found
here}\label{the-iac-file-that-created-the-webapp-can-be-found-here}

\begin{itemize}
\tightlist
\item[$\square$]
  webapps.tf (.tf for terraform)
\end{itemize}

https://github.com/setrar/Clouds/blob/main/Labs/Azure/lab1/IaC/webapps.tf

\begin{itemize}
\tightlist
\item[$\square$]
  The github repo containing the APS.NET content can be found here
\end{itemize}

https://github.com/setrar/CloudsASPXContent

    \section{Deploy static site on Azure Blob
Store}\label{deploy-static-site-on-azure-blob-store}

    \(\color{blue}\textbf{What is the public address of the Blob store static site?}\)

\begin{center}\rule{0.5\linewidth}{0.5pt}\end{center}

\(\color{blue} \textbf{Your answer:}\)

https://blobstoreclouds2025eurbr.z13.web.core.windows.net

\paragraph{The IaC file that created the blob store can be found
here}\label{the-iac-file-that-created-the-blob-store-can-be-found-here}

\begin{itemize}
\tightlist
\item[$\square$]
  blockstores.tf (.tf for terraform)
\end{itemize}

https://github.com/setrar/Clouds/blob/main/Labs/Azure/lab1/IaC/blockstores.tf

    \begin{center}\rule{0.5\linewidth}{0.5pt}\end{center}

\(\Large{\textbf{Summary Questions. }}\)

    \(\color{blue}\textbf{You should have created a RG for each assignment separately.}\)
\(\color{blue}\textbf{Report the cost distribution across various RGs that you can find in }\)
\(\color{blue}\textbf{Azure Cost Analysis under Cost Management.}\)

\begin{center}\rule{0.5\linewidth}{0.5pt}\end{center}

\(\color{blue} \textbf{Your answer:}\)

For the entire lab (Assignement 1), one resource group was used
\texttt{lab1-resources}

€2.20

IaC (Infrastructure as Code), using \texttt{OpenTofu} helps saving cost
by creating all the infrastructure at once and tearing it down all
together when needed. Some extra commands needed to applied manually but
in general running the scripts took about 1 minute to launch.

\begin{figure}
\centering
\pandocbounded{\includegraphics[keepaspectratio]{images/lab1-cost-analysis.png}}
\caption{image}
\end{figure}


    % Add a bibliography block to the postdoc
    
    
    
\end{document}
